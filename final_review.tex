%\documentstyle[11pt,a4]{article}
%\documentclass[a4paper]{article}
\documentclass[a4paper, 12pt]{article}
% Seems like it does not support 9pt and less. Anyways I should stick to 10pt.
%\documentclass[a4paper, 9pt]{article}
\topmargin-2.0cm

\usepackage{fancyhdr}
\usepackage{pagecounting}
\usepackage[dvips]{color}

% Color Information from - http://www-h.eng.cam.ac.uk/help/tpl/textprocessing/latex_advanced/node13.html

% NEW COMMAND
% marginsize{left}{right}{top}{bottom}:
%\marginsize{3cm}{2cm}{1cm}{1cm}
%\marginsize{0.85in}{0.85in}{0.625in}{0.625in}

%\advance\oddsidemargin-0.65in
%\advance\evensidemargin-1.5cm
\textheight9.2in
\textwidth6.75in
\newcommand\bb[1]{\mbox{\em #1}}
\def\baselinestretch{1.5}
%\pagestyle{empty}

\newcommand{\hsp}{\hspace*{\parindent}}
\definecolor{gray}{rgb}{0.4,0.4,0.4}
%\definecolor{gray}{rgb}{1.0,1.0,1.0}


\begin{document}
\thispagestyle{fancy}
%\pagenumbering{gobble}
%\fancyhead[location]{text} 
% Leave Left and Right Header empty.
\lhead{}
\rhead{}
%\rhead{\thepage}
\renewcommand{\headrulewidth}{0pt} 
\renewcommand{\footrulewidth}{0pt} 
\rhead{\textcolor{gray}{\thepage/\totalpages{}}}
%\vspace*{0.1cm}
\begin{center}
{\LARGE \bf Final Review}\\
\vspace*{0.1cm}
{\LARGE \bf A Mobile Platform for Adaptive Grocery Tracking}\\
\vspace*{0.1cm}

{\bf Ishaan Biswas and Rishanka Prabhu and Mohammed Ameen Maleej}\\

\vspace*{0.1cm}
{\normalsize Review from : Ashiwan Sivakumar (asivakum@purdue.edu)}

\noindent$\bullet$ Overview:
The paper presents an application that tracks users grocery receipts by performing OCR on scanned
images of the receipts and applies a linear regression model to predict the lifetime of items purchased
from the old date of purchases, thereby notifies users of probable depletion of items. The application also
suggests recipes based on items available in the users pantry. The authors propose fuzzy string matching to augment
and improve the accuracy of feature extraction of the OCR module. The evaluation shows string matching improves the accuracy
of OCR and the linear regression model works reasonably well for a single user over 6 weeks period. 

\noindent$\bullet$ Strengths:
\begin{itemize}
\item The problem is important and the application is interesting.
\item The authors have taken efforts to minimize user interaction which is great. 
\item The feature of predicting usage patterns and suggesting recipes is cool.
\end{itemize}

\noindent$\bullet$ Weaknesses:
\begin{itemize}
\item Even though the application aims at building an unified platform for grocery tracking, it feels like it is an amalgamation of features from popular apps - grocery management and recipe suggestion. 
\item I have questions regarding {\bf the memory usage, computation time and scalability} of the fuzzy string matching algorithm. {\bf The generic problem of context-based string matching is known to be NP-complete.}. I understand that you have a pre-generated database of strings. But still I believe, the paper should have justification as to whether the computation time is tolerable. 
\item The main aim of the application is to reduce user intervention as much as possible. This is one of the main differences from existing apps. But to improve the accuracy of the OCR and the prediction model, the application requires some manual intervention. Is this not contradicting the motivation of the application? A quantification of user intervention in Alfredo with existing apps would help alleviate this question on reviewer's mind.
\item The simple linear regression model used with predictor variables (X) as the actual lifetime of an item observed from history and output variable (Y) as the expected lifetime, has many assumptions. First I am not sure how to interpret that the expected lifetime of an item would depend on the actual lifetime from observed history. Won't observed history itself depend on factors like user's cuisine interest, users cooking pattern, time at which user decides to buy other items and so many? There are many dependencies between factors here and hence such a simple model might not work in practice. Also applicable to any linear regression, it assumes the linearity between Y and X, constant variance. Not sure how far this is true for the X and Y considered here. Your evaluation results show some anamolies in figures 5a-5d. Please see "comments to authors" for more details.
\item There is no summary on the accuracy of the linear regression and the prediction error (something like Root mean square error). Why pick only 4 items and show the accuracy? Can't you show the accuracy for all items in a user's receipt over the 6 week period when you are already doing it for 4? Showing only 4 example items does not convince the reviewer of the validity and the general applicability of the results.
\item There is no comparison with existing apps. Would make the paper more stronger.  
\end{itemize}

\noindent$\bullet$ Comments to authors:
\begin{itemize}
\item Figures 5b, 5d have more predicted values than actual values? Are there overlapping points that we don't see equal number of actual and predicted? Or is there an error in the graphs?
\item In figures 5a and 5c, once the predicted model starts closely following the actual lifetime, the prediction is always lower than the actual. Is this because of the penalty that you are adding to the model to give more importance to recently observed data points?
\item Can you explain why in 5c, for the 5th week the predicted value is so low compared to actual? This does not convince me of the usage of a linear regression model.
\item Some minor comments - Section 3.2. "used to for" --> "used to". The defacto and canonical "to compare to strings" --> "to compare two strings??". 
\item There is mention of Table X in section 3.3. I don't see any Table X in the paper.

\end{itemize}
\end{center}
%\vspace*{0.2cm}

\end{document}

